\documentclass[twoside, titlepage]{amsart}

\newcommand{\authorName}{Clark Zinzow}  % Author's name.
\newcommand{\bookName}{Booky McBookface}  % Book name.
\newcommand{\bookEdition}{1st}
\newcommand{\sectionType}{Chapter} % Type of section (chapter, section, etc.)
\newcommand{\sectionNumber}{1} % Chapter or section that problems are from.
\newcommand{\startDate}{\today}
\newcommand{\bookAuthor}{Johnny Authorman}  % Book author's name.
\newcommand{\universityName}{University of Wisconsin-Madison}
\newcommand{\authorEmail}{czinzow@wisc.edu}  % Author's e-mail.

% Basic amsthm definitions

% Styles:
%   - plain:  italic text, extra space above and below;
%   - definition:  upright text, extra space above and below;
%   - remark:  upright text, no extra space above and below.

% plain style
\theoremstyle{plain} % default
\newtheorem{thm}{Theorem}[section]
\newtheorem{prop}[thm]{Proposition}
\newtheorem{conj}[thm]{Conjecture}
\newtheorem{crit}[thm]{Criterion}
\newtheorem{assrt}[thm]{Assertion}
\newtheorem{lem}[thm]{Lemma}
\newtheorem*{cor}{Corollary}
\newtheorem{fact}[thm]{Fact}

% definition style
\theoremstyle{definition}
\newtheorem{defn}{Definition}[section]
\newtheorem{condition}{Condition}[section]
\newtheorem{problem}{Problem}[section]
\newtheorem{axiom}{Axiom}[section]
\newtheorem{exmp}{Example}[section]
\newtheorem{property}{Property}[section]
\newtheorem{assump}{Assumption}[section]
\newtheorem{hypoth}{Hypothesis}[section]
\newtheorem{ques}{Question}[section]
\newtheorem{xca}[exmp]{Exercise}

% remark style
\theoremstyle{remark}
\newtheorem*{rem}{Remark}
\newtheorem*{note}{Note}
\newtheorem*{clm}{Claim}
\newtheorem*{summary}{Summary}
\newtheorem*{acknowl}{Acknowledgments}
\newtheorem*{concl}{Conclusion}
\newtheorem*{case}{Case}

\newtheoremstyle{notation}
{}
{}
{}
{}
{\itshape}
{:}
{\newline}
{}
\theoremstyle{notation}
\newtheorem*{notation}{Notation}

% Proof workaround:  The proof environment is defined as a LaTeX list, so a 
% substitute ``name" that is longer than one output line will not break as it
% should.  The following is a preamble hack until amsthm fixes this.
\makeatletter
\renewenvironment{proof}[1][\proofname]{\par
	\pushQED{\qed}%
	\normalfont \topsep6\p@\@plus6\p@\relax
	\trivlist
	\item\relax
	{\itshape
		#1\@addpunct{.}}\hspace\labelsep\ignorespaces
}{%
\popQED\endtrivlist\@endpefalse
}
\makeatother

\usepackage{mathrsfs}
\usepackage{amssymb,amsmath,amsthm}
\usepackage{enumerate}
\usepackage[margin=1in]{geometry}
\usepackage{verbatim}
\usepackage[section]{placeins}
\usepackage{array}
\usepackage{nicefrac}
\usepackage{listings}
\usepackage{parskip}
\usepackage{graphicx}  % Note:  Use \afterpage{\clearpage} to flush all
%        processed floats, not \clearpage.
\usepackage[colorinlistoftodos]{todonotes}
\usepackage{tikz-cd}
\usepackage{bm}
\usepackage{dsfont}
\usepackage{mdwlist}
\usepackage{underscore}
\usepackage[colorlinks=true, citecolor=red, urlcolor=blue]{hyperref}

% Todo notes commands.
\newcommand{\unfinished}[1]{\todo[inline, color=red!40]{#1}}
\newcommand{\argcheck}[1]{\todo[inline, color=orange!40]{#1}}
\newcommand{\needprove}[1]{\todo[inline, color=violet!40]{#1}}
\newcommand{\detail}[1]{\todo[inline, color=blue!40]{#1}}
\newcommand{\insertref}[1]{\todo[inline, color=green!40]{#1}}
\newcommand{\badnotation}[1]{\todo[inline, color=yellow!40]{#1}}
\newcommand{\further}[1]{\todo[inline, color=cyan!40]{#1}}

\newcommand{\ses}[3]{%
	\begin{tikzcd}[ampersand replacement=\&]
		0 \arrow[r] \& #1 \arrow[r] \& #2 \arrow[r] \& #3 \arrow[r] \& 0
	\end{tikzcd}%
}
\newcommand{\sesm}[5]{%
	\begin{tikzcd}[ampersand replacement=\&]
		0 \arrow[r] \& #1 \arrow[r, "#4"] \& #2 \arrow[r, "#5"] \& #3 \arrow[r] \& 0
	\end{tikzcd}%
}
\newcommand{\suchthat}{\: \big\rvert \:}
\newcommand{\suchthatBig}{\: \Big\rvert \:}
\newcommand{\suchthathuge}{\: \huge\rvert \:}

% Subject specific commands.



\begin{document}
	
	\begin{titlepage}
		\centering
		\vspace*{\baselineskip}
		\rule{\textwidth}{1.6pt}\vspace*{-\baselineskip}\vspace*{2pt}
		\rule{\textwidth}{0.4pt}\\[\baselineskip]
		{\Huge Solutions to Exercises from \\ \sectionType \:\sectionNumber \:of \\[0.2\baselineskip] \emph{\bookName}
		}\\[0.2\baselineskip]
		\rule{\textwidth}{0.4pt}\vspace*{-\baselineskip}\vspace*{3pt}
		\rule{\textwidth}{1.6pt}\\[\baselineskip]
		\vspace*{3\baselineskip}
		\huge  \par
		\vspace*{\baselineskip}
		{\itshape \universityName \par}
		\vspace*{3\baselineskip}
		\huge \textbf{Author of Solutions:} { \authorName} \\
		\vspace*{\baselineskip}
		\huge \textbf{Date Edited:} { \startDate } \par
		\vspace*{\baselineskip}
		\huge \textbf{Text:} { \emph{\bookName,  \bookEdition \:Ed.}} \\
		\vspace*{\baselineskip}
		\huge \textbf{Author of Text:} { \bookAuthor } \par
		\vspace*{\baselineskip}
		\huge \textbf{Contact:} { \authorEmail } \\
	\end{titlepage}
	
	\makeatletter
	\providecommand\@dotsep{5}
	\makeatother
	\listoftodos\relax
	
	\clearpage
	
	This is a set of solutions, created by \authorName, to the \sectionType \:\sectionNumber \:exercises from the text \emph{\bookName, \bookEdition \:Ed.} by \bookAuthor.
	\hspace{0pt} \\
	
	\section*{Notation:}
	We use the following non-standard notation:
	\begin{itemize}
		\item 
	\end{itemize}
	
	\section*{Acknowledgments:}
	Thank you to the following people for helping me, in some way, with these solutions:
	\begin{itemize}
		\item 
	\end{itemize}
	
	\clearpage
	
	\section*{Exercise 1}
	\stepcounter{section}
	
		
	
	\begin{proof}
		
		
		
	\end{proof}
	
	\clearpage
	
	\section*{Exercise 2}
	\stepcounter{section}
	
		
	
	\begin{proof}
		
		
		
	\end{proof}
	
	\clearpage
	
	% Alpha bibliography style.  Make second argument to thebigliography environment the largest largest explicit label used.
	\begin{thebibliography}{"Graf08"}
		\bibitem[Fol99]{folland99}
		Gerald B. Folland,
		\emph{Real Analysis: Modern Techniques and Their Applications},
		Wiley,
		2nd Edition,
		1999.
		\bibitem[Graf08]{grafakos08}
		Loukas Grafakos,
		\emph{Classical Fourier Analysis},
		Springer,
		2nd Edition,
		2008.
		\bibitem[Rud86]{rudin86}
		Walter Rudin,
		\emph{Real and Complex Analysis},
		McGraw-Hill Education,
		3rd Edition,
		1986.
		\bibitem[Rud91]{rudin91}
		Walter Rudin,
		\emph{Functional Analysis},
		McGraw-Hill Science/Engineering/Math,
		2nd Edition,
		1991.
		\bibitem[HR63]{hewittross63}
		Edwin Hewitt and Kenneth A. Ross,
		\emph{Abstract Harmonic Analysis},
		Springer,
		1st Edition,
		1963.
	\end{thebibliography}
	
\end{document}
