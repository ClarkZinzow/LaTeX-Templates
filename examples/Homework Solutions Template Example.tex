\documentclass[twoside, titlepage]{amsart}

% Insert your details here!
\newcommand{\authorName}{Clark Zinzow}  % Author's name.
\newcommand{\homeworkNumber}{1}  % Homework number.
\newcommand{\subject}{Subject}  % Subject.
\newcommand{\courseNumber}{101}  % Course number.
\newcommand{\courseName}{Introduction to Intermediate}  % Course name.
\newcommand{\instructor}{Professor Professorson}  % Name of lecturer.
\newcommand{\submitDate}{\today}  % Date solutions last edited.
\newcommand{\universityName}{University of Wisconsin-Madison}  % University.
\newcommand{\authorEmail}{czinzow@wisc.edu}  % Author's e-mail.

% Basic amsthm definitions

% Styles:
%   - plain:  italic text, extra space above and below;
%   - definition:  upright text, extra space above and below;
%   - remark:  upright text, no extra space above and below.

% plain style
\theoremstyle{plain} % default
\newtheorem{thm}{Theorem}[section]
\newtheorem{prop}[thm]{Proposition}
\newtheorem{conj}[thm]{Conjecture}
\newtheorem{crit}[thm]{Criterion}
\newtheorem{assrt}[thm]{Assertion}
\newtheorem{lem}[thm]{Lemma}
\newtheorem*{cor}{Corollary}
\newtheorem{fact}[thm]{Fact}

% definition style
\theoremstyle{definition}
\newtheorem{defn}{Definition}[section]
\newtheorem{condition}{Condition}[section]
\newtheorem{problem}{Problem}[section]
\newtheorem{axiom}{Axiom}[section]
\newtheorem{exmp}{Example}[section]
\newtheorem{property}{Property}[section]
\newtheorem{assump}{Assumption}[section]
\newtheorem{hypoth}{Hypothesis}[section]
\newtheorem{ques}{Question}[section]
\newtheorem{xca}[exmp]{Exercise}

% remark style
\theoremstyle{remark}
\newtheorem*{rem}{Remark}
\newtheorem*{note}{Note}
\newtheorem*{clm}{Claim}
\newtheorem*{summary}{Summary}
\newtheorem*{acknowl}{Acknowledgments}
\newtheorem*{concl}{Conclusion}
\newtheorem*{case}{Case}

\newtheoremstyle{notation}
{}
{}
{}
{}
{\itshape}
{:}
{\newline}
{}
\theoremstyle{notation}
\newtheorem*{notation}{Notation}

% Proof workaround:  The proof environment is defined as a LaTeX list, so a 
% substitute ``name" that is longer than one output line will not break as it
% should.  The following is a preamble hack until amsthm fixes this.
\makeatletter
\renewenvironment{proof}[1][\proofname]{\par
	\pushQED{\qed}%
	\normalfont \topsep6\p@\@plus6\p@\relax
	\trivlist
	\item\relax
	{\itshape
		#1\@addpunct{.}}\hspace\labelsep\ignorespaces
}{%
\popQED\endtrivlist\@endpefalse
}
\makeatother

% Packages
\usepackage{mathrsfs}
\usepackage{amssymb,amsmath,amsthm}
\usepackage{enumerate}
\usepackage[margin=1in]{geometry}
\usepackage{verbatim}
\usepackage[section]{placeins}
\usepackage{array}
\usepackage{nicefrac}
\usepackage{parskip}
\usepackage{graphicx}  % Note:  Use \afterpage{\clearpage} to flush all
%        processed floats, not \clearpage.
\usepackage[colorinlistoftodos]{todonotes}
\usepackage{fancyhdr}
\usepackage{dsfont}
\usepackage{mdwlist}
\usepackage{underscore}
\usepackage[colorlinks=true, citecolor=red, urlcolor=blue]{hyperref}

% Subject specific packages
\usepackage{tikz-cd}  % Great for commutative diagrams.
\usepackage{bm}  % Easy bold vectors.
\usepackage{listings}  % Great for source code.

% Todo notes commands, color-coded to type of todo.
\newcommand{\unfinished}[1]{\todo[inline, color=red!40]{#1}}
\newcommand{\argcheck}[1]{\todo[inline, color=orange!40]{#1}}
\newcommand{\needprove}[1]{\todo[inline, color=violet!40]{#1}}
\newcommand{\detail}[1]{\todo[inline, color=blue!40]{#1}}
\newcommand{\insertref}[1]{\todo[inline, color=green!40]{#1}}
\newcommand{\badnotation}[1]{\todo[inline, color=yellow!40]{#1}}
\newcommand{\further}[1]{\todo[inline, color=cyan!40]{#1}}

% Such that symbol command.
\newcommand{\suchthat}{\: \big\rvert \:}
\newcommand{\suchthatBig}{\: \Big\rvert \:}
\newcommand{\suchthathuge}{\: \huge\rvert \:}

% Subject specific commands:

\newcommand{\morph}{\longrightarrow}
\newcommand{\lmorph}{\longleftarrow}
\newcommand{\morpha}[1]{\overset{#1}{\morph}}
\newcommand{\lmorpha}[1]{\overset{#1}{\lmorph}}
\newcommand{\ses}[3]{%
	\begin{tikzcd}[ampersand replacement=\&]
		0 \arrow[r] \& #1 \arrow[r] \& #2 \arrow[r] \& #3 \arrow[r] \& 0
	\end{tikzcd}%
}
\newcommand{\sesm}[5]{%
	\begin{tikzcd}[ampersand replacement=\&]
		0 \arrow[r] \& #1 \arrow[r, "#4"] \& #2 \arrow[r, "#5"] \& #3 \arrow[r] \& 0
	\end{tikzcd}%
}
\newcommand{\simeqw}{\stackrel{\mathcal{\normalfont\mbox{w}}}{\simeq}}
\newcommand{\id}{\textbf{1}}
\newcommand{\idmap}{\text{id}}
\newcommand{\realproj}{\mathbb{R} \text{P}}
\newcommand{\rank}{\text{rank} \:}
\newcommand{\grd}{\bullet}
\newcommand{\im}{\text{Im} \:}
\newcommand{\coker}{\text{coker} \:}
\newcommand{\CW}{\text{CW}}
\newcommand{\redhom}[1]{\widetilde{#1}}
\newcommand{\interior}[1]{\text{int}(#1)}
\newcommand{\inv}[1]{\overline{#1}}

\begin{document}

	\begin{titlepage}
		\centering
		\vspace*{\baselineskip}
		\rule{\textwidth}{1.6pt}\vspace*{-\baselineskip}\vspace*{2pt}
		\rule{\textwidth}{0.4pt}\\[\baselineskip]
		{\Huge Solutions to Homework \homeworkNumber \\[0.2\baselineskip]
		for \\[0.4\baselineskip]
		\subject\:\courseNumber: \courseName
		}\\[0.2\baselineskip]
		\rule{\textwidth}{0.4pt}\vspace*{-\baselineskip}\vspace*{3pt}
		\rule{\textwidth}{1.6pt}\\[\baselineskip]
		\vspace*{3\baselineskip}
		\huge  \par
		\vspace*{\baselineskip}
		{\itshape \universityName \par}
		\vspace*{3\baselineskip}
		\huge \textbf{Author of Solutions:} { \authorName} \\
		\vspace*{\baselineskip}
		\huge \textbf{Instructor:} { \instructor} \\
		\vspace*{\baselineskip}
		\huge \textbf{Date Submitted:} { \submitDate } \par
		\vspace*{\baselineskip}
		\huge \textbf{Contact:} { \authorEmail} \\
		\vspace*{\baselineskip}
	\end{titlepage}
	
	\makeatletter
	\providecommand\@dotsep{5}
	\makeatother
	\listoftodos\relax
	
	\clearpage
	
	This is a set of solutions, created by \authorName, to Homework \homeworkNumber\:for \subject\:\courseNumber: \courseName.
	\hspace{0pt} \\

	\section*{Notation:}
	To avoid any possible ambiguity or confusion, we make special note of our use of the following notation:
	\begin{itemize}
		\item Let $\mathbb{Z}_+$ denote the set of positive integers.
		\item Given two paths $f$ and $g$, let $f \sim g$ suggest that $f$ is homotopic to $g$ (and vice versa.)
		\item Let $\textbf{1}$ denote the group identity element (unless specified otherwise.)
		\item For two groups $G$ and $H$, let $G \cong H$ denote $G$ being isomorphic to $H$.
		\item We frequently represent the fact that two groups (spaces) are isomorphic (homeomorphic) by equality.
		\item For $x \in \mathbb{R}^n$ and $\gamma \in \mathbb{R}$, let $x \preceq \gamma$ denote $x_i \leq \gamma$ for $i = 1, \dots, n$, and define $\prec$, $\succ$, and $\succeq$ analogously.
		\item For any set $X$, let $\interior{X}$ denote the interior of the set $X$.
		\item For any loop (path) $\gamma$, let $\inv{\gamma}$ represent the inverse loop (path).  I.e., $\inv{\gamma} = \gamma^{-1}$.
	\end{itemize}
	
	\section*{Acknowledgments:}
	Thank you to the following people for helping me, in some way, with these solutions:
	\begin{itemize}
		\item Allen Hatcher, whose Algebraic Topology textbook, \cite{hatcher01}, is awesome.
	\end{itemize}
	

	\clearpage
	
	\section*{Exercise 1}
	\stepcounter{section}
	
	The mapping torus $T_f$ of a map $f: X \to X$ is the quotient $X \times I$
	$$T_f = \frac{X \times I}{(x,0) \sim (f(x),1)}.$$
	Let $A$ and $B$ be copies of $S^1$, let $X = A \vee B$, and let $p$ be the wedge point of $X$.  Let $f: X \to X$ be a map that satisfies $f(p) = p$, carries $A$ into $A$ by a degree-3 map, and carries $B$ into $B$ by a degree-5 map.
	
	\begin{enumerate}[(a)]
		\item Equip $T_f$ with a CW structure by attaching cells to $X \vee S^1$.
		\item Compute a presentation of $\pi_1(T_f)$. 
		\item Compute $H_1(T_f,K)$.
	\end{enumerate}
	
	\begin{proof}
		
		\begin{enumerate}[(a)]
			\item Let $X = S^1 \vee S^1$.  In the typical, explicit formulation of the mapping torus, we would have
			$$T_f = \frac{(S^1 \vee S^1) \times I}{(x,0) \sim (f(x),1)}.$$
			I.e., $S^1 \vee S^1 \times \{0\}$ is identified with $S^1 \vee S^1 \times \{1\}$.  Given that $f$ preserves the basepoint $p$, this identification can be seen as a loop homeomorphic to $S^1$ attached to $S^1 \vee S^1$ at $p$, yielding $X \vee S^1 \subset T_f$.
			
			Now, note that $X \vee S^1$ is generated by the loops $a,b,c$, with $c$ being the loop of the newly wedged copy of $S^1$.  The mapping torus can be formed by attaching a 2-cell around the path $a c \inv{f}_*(a) \inv{c}$ and attaching a 2-cell around the path $b c \inv{f}_*(b) \inv{c}$.
			
			\unfinished{Include a representative picture here?}
			\item Given that
			$$\pi_1(X \vee S^1) = \pi_1(S^1 \vee S^1 \vee S^1) = \langle a,b,c \rangle,$$
			by Proposition 1.26 of \cite{hatcher01}, we have that
			$$\pi_1(T_f) = \langle a,b,c \suchthat a c \inv{f}_*(a) \inv{c}, \: b c \inv{f}_*(b) \inv{c}\rangle.$$
			\item Let $H_1(T_f)$ denote $H_1(T_f,K)$ in the following argument.  Note that we have established that $T_f$ has the following CW structure:  one 0-cell (the basepoint wedge point of $X$, $p$), three 1-cells $\{a,b,c\}$, and two 2-cells attached by the words $a c \inv{f}_*(a) \inv{c}$ and $b c \inv{f}_*(b) \inv{c}$.  Recall that a cellular chain complex of a CW complex $X$ is given by
			$$\begin{tikzcd}
			\cdots \arrow[r] & H_{n+1}(X^{n+1},X^n) \arrow[r, "d_{n+1}"] & H_n(X^n,X^{n-1}) \arrow[r, "d_n"] & H_{n-1}(X^{n-1},X^{n-2}) \arrow[r] & \cdots
			\end{tikzcd}$$
			Moreover, via Lemma 2.34 of \cite{hatcher01}, we have the following diagram
			$$\begin{tikzcd}[column sep=tiny]
			& [1.5em] & & & 0 & & [1.5em] \\
			& 0 \arrow[dr] & & H_n(X^{n+1}) \cong H_n(X) \arrow[ur] & & & \\
			& & H_n(X^n) \arrow[ur] \arrow[dr, "j_n"] & & & & \\
			\cdots \arrow[r] & H_{n+1}(X^{n+1},X^n) \arrow[ur, "\partial_{n+1}"] \arrow[rr, "d_{n+1}"] & & H_n(X^n,X^{n-1}) \arrow[rr, "d_n"] \arrow[dr, "\partial_n"] & & H_{n-1}(X^{n-1},X^{n-2}) \arrow[r] & \cdots \\
			& & & & H_{n-1}(X^{n-1}) \arrow[ur, "j_{n-1}"] & & \\
			& & & 0 \arrow[ur] & & &
			\end{tikzcd}$$
			By this diagram and Theorem 2.35 of \cite{hatcher01}, we have that
			$$H_n(X) \cong \ker d_n / \im d_{n+1}.$$
			
			This suggests that our particular area of interest in this cellular chain complex is
			$$\begin{tikzcd}
			\cdots \arrow[r] & H_3(X^3,X^2) \arrow[r, "d_3"] & H_2(X^2,X^1) \arrow[r, "d_2"] & H_1(X^1,X^0) \arrow[r, "d_1"] & H_0(X^0) \arrow[r, "d_0"] & 0
			\end{tikzcd}$$
			If we are able to compute $\ker d_1$ and $\im d_2$, we will have $H_1(X)$.
			
			By our construction of a CW complex for $T_f$, we have that $T_f$ has one 0-cell, three 1-cells, and two 2-cells, and given that $\mathcal{C}_n = H_n(X^n, X^{n-1}) = \mathbb{Z}^{\text{\# } n \text{-cells}}$, we have that the cellular chain complex of $T_f$ is
			$$\begin{tikzcd}
			0 \arrow[r] & \mathbb{Z}^2 \arrow[r, "d_2"] & \mathbb{Z}^3 \arrow[r, "d_1"] & \mathbb{Z} \arrow[r] & 0
			\end{tikzcd}$$
			where $H_3(X^3,X^2) = 0$ because $X$ has an empty $k$-skeleton for $k > 2$.  We clearly have that $T_f$ is connected and only has one 0-cell, which suggests that $d_1 = 0$, hence $\ker d_1 = \mathbb{Z}^3$.  Therefore, all that remains is to calculate $d_2$.
			
			By the Cellular Boundary Formula,
			$$d_n(e_\alpha^n) = \sum_\beta d_{\alpha \beta} e_\beta^{n-1}$$
			where $d_{\alpha \beta}$ is the degree of the map $S_\alpha^{n-1} \to X^{n-1} \to S_\beta^{n-1}$ that is the composition of the attaching map of $e_\alpha^n$ with the quotient map collapsing $X^{n-1} - e_\beta^{n-1}$ to a point.  Let $e_a^1$, $e_b^1$, and $e_c^1$ denote the 1-cells corresponding to the loop $a$, $b$, and $c$, respectively.  Let $e_1^2$ denote the 2-cell attached around the path $a c \inv{f}_*(a) \inv{c}$ and let $e_2^2$ denote the 2-cell attached around the path $b c \inv{f}_*(b) \inv{c}$.  Then we have that
			\begin{align*}
			d_2(e_1^2) & = \sum_\beta d_{1 \beta} \cdot e_\beta^1 \\
			& = d_{1 a} \cdot e_a^1 + d_{1 b} \cdot e_b^1 + d_{1 c} \cdot e_c^1 \\
			& = (a - f_*(a)) + (0) + (c - c) \\
			& = a - f_*(a) \\
			& = a - 3 \cdot a \\
			& = -2 a
			\end{align*}
			by the fact that $f$ carries $A$ into $A$ by a degree-3 map.  Proceeding similarly for the other 2-cell, we have that
			\begin{align*}
			d_2(e_2^2) & = \sum_\beta d_{2 \beta} \cdot e_\beta^1 \\
			& = d_{2 a} \cdot e_a^1 + d_{2 b} \cdot e_b^1 + d_{2 c} \cdot e_c^1 \\
			& = (0) + (b - f_*(b)) + (c - c) \\
			& = b - f_*(b) \\
			& = b - 5 \cdot b \\
			& = -4 b
			\end{align*}
			by the fact that $f$ carries $B$ into $B$ by a degree-5 map.
			
			Therefore, we have that
			$$d(\id) = \begin{bmatrix}
			-2 \\
			-4 \\
			0
			\end{bmatrix}$$
			hence
			$$\im d = (-2) \mathbb{Z} \times (-4) \mathbb{Z} \times \{\id\} = 2 \mathbb{Z} \times 4\mathbb{Z} \times \{\id\}.$$
			Therefore,
			$$H_1(T_f) = \mathbb{Z}^3 / 2 \mathbb{Z} \times 4 \mathbb{Z} \times \{\id\} = \mathbb{Z} / 2\mathbb{Z} \oplus \mathbb{Z} / 4\mathbb{Z} \oplus \mathbb{Z} = \mathbb{Z}_2 \oplus \mathbb{Z}_4 \oplus \mathbb{Z}.$$
			
			\textbf{Note:}
			
			This could also be seen by considering an exact sequence similar to the Mayer-Vietoris sequence, namely
			$$\begin{tikzcd}
			\cdots \arrow[r] & H_n(X) \arrow[r, "\id - f_*"] & H_n(X) \arrow[r, "\iota_*"] & H_n(T_f) \arrow[r] & H_{n-1}(X) \arrow[r] & \cdots
			\end{tikzcd}$$
			with $\iota: X \hookrightarrow T_f$ being the inclusion.  The derivation of a more general form of this exact sequence is given in Example 2.48 of \cite{hatcher01}, with this explicit exact sequence given in Exercise 2.2.30 of \cite{hatcher01}.
			
			\detail{Delete this note, or maybe include more detail about alternative method?}
			
			By Corollary 2.25 of \cite{hatcher01}, we have that
			$$H_n(X) = H_n(A \vee B) \cong H_n(A) \oplus H_n(B)$$
			for $n > 0$, which, when coupled with the facts that $H_n(S^1) = 0$ for $n > 1$ and $H_2(T_f) = 0$, gives us the exact sequence
			$$\begin{tikzcd}
			0 \arrow[r] & H_1(A) \oplus H_1(B) \arrow[r, "\id - f_*"] \arrow[d, "\cong"] & H_1(A) \oplus H_1(B) \arrow[r, "\iota_*"] \arrow[d, "\cong"] & H_1(T_f) \arrow[r] & H_0(A \vee B) \arrow[r, "\id - f_*"] \arrow[d, "\cong"] & H_0(A \vee B) \arrow[r] \arrow[d, "\cong"] & 0 \\
			& \mathbb{Z} \oplus \mathbb{Z} \arrow[r, "\id - f_*"] & \mathbb{Z} \oplus \mathbb{Z} & & \mathbb{Z} \arrow[r, "\id - f_*"] & \mathbb{Z} &
			\end{tikzcd}$$
			Since $f$ takes $A$ into $A$ by a degree-3 map and takes $B$ into $B$ by a degree-5 map, we have that
			$$\begin{tikzcd}
			0 \arrow[r] & H_1(A) \oplus H_1(B) \arrow[r, "-2 \text{ on } A", "-4 \text{ on } B" below] \arrow[d, "\cong"] & H_1(A) \oplus H_1(B) \arrow[r, "\iota_*"] \arrow[d, "\cong"] & H_1(T_f) \arrow[r] & H_0(A \vee B) \arrow[r, "0"] \arrow[d, "\cong"] & H_0(A \vee B) \arrow[r] \arrow[d, "\cong"] & 0 \\
			& \mathbb{Z} \oplus \mathbb{Z} \arrow[r, "-2 \text{ on } A", "-4 \text{ on } B" below] & \mathbb{Z} \oplus \mathbb{Z} & & \mathbb{Z} \arrow[r, "0"] & \mathbb{Z} &
			\end{tikzcd}$$
			This gives us the exact sequence
			$$\begin{tikzcd}
			0 \arrow[r] & \mathbb{Z}_2 \oplus \mathbb{Z}_4 \arrow[r] & H_1(T_f) \arrow[r] & \mathbb{Z} \arrow[r] & 0
			\end{tikzcd}$$
			Given that $\mathbb{Z}$ is free, we have that this exact sequence splits. Therefore, bv the Splitting Lemma, we have that
			$$H_1(T_f) = \mathbb{Z}_2 \oplus \mathbb{Z}_4 \oplus \mathbb{Z}.$$
			\insertref{Should we cite the Splitting Lemma from the text?}
		\end{enumerate}
		
	\end{proof}
	
	\clearpage
	
	\section*{Exercise 2}
	\stepcounter{section}
	
	Use Mayer-Vietoris to compute the degree of the suspension $\Sigma f$ of a map $f: S^n \to S^n$.
	
	\begin{proof}
		
		Let
		$$A = (S^n \times I) / (S^n \times 1), \hspace{6mm} B = (S^n \times I) / (S^n \times 0).$$
		I.e., $A$ is the upper cone of $S^{n+1}$ and $B$ is the lower cone of $S^{n+1}$.  Therefore, $A, B \subset S^{n+1}$, $S^{n+1} = \interior{A} \cup \interior{B}$, and $S^n = A \cap B$.  We therefore have the reduced Mayer-Vietoris sequence
		$$\begin{tikzcd}
		\cdots \arrow[r] & \redhom{H}_{n+1}(S^n) \arrow[r, "\Phi"] & \redhom{H}_{n+1}(A) \oplus \redhom{H}_{n+1}(B) \arrow[r, "\Psi"] & \redhom{H}_{n+1}(S^{n+1}) \arrow[r, "\partial"] & \redhom{H}_n(S^n) \arrow[r] & \cdots
		\end{tikzcd}$$
		Since $A$ and $B$ are both contractible, we have that
		$$\redhom{H}_i(A) \oplus \redhom{H}_i(B) \cong 0$$
		for $i = 0, \dots, n$.  Letting $\partial = \partial_{n+1}$, the above reduced Mayer-Vietoris sequence is equivalent to the exact sequence
		$$\begin{tikzcd}
		\cdots \arrow[r, "\Phi_{n+1}"] & 0 \arrow[r, "\Psi_{n+1}"] & \redhom{H}_{n+1}(S^{n+1}) \arrow[r, "\partial"] & \redhom{H}_n(S^n) \arrow[r, "\Phi_n"] & 0 \arrow[r, "\Psi_n"] & \cdots
		\end{tikzcd}$$
		Therefore, by (iii) on page 114 of \cite{hatcher01}, we have that $\partial$ is an isomorphism hence $\redhom{H}_{n+1}(S^{n+1}) \cong \redhom{H}_n(S^n)$.
		
		Note that $f$ induces a map $Af: (A, S^n) \to (A, S^n)$ with quotient $\Sigma f: (A, S^n) \to (A/S^n,S^n/S^n)$.  Given that $S^n \subset A$ and there clearly exists a neighborhood $U$ of $S^n$ in $A$ that deformation retracts onto $S^n$, by Proposition 2.22 of \cite{hatcher01} we have that $\redhom{H}_{n+1}(A, S^n) \cong \redhom{H}_{n+1}(A / S^n)$.
		
		We therefore have the following $(A,S^n)$ pair long exact sequence:
		$$\begin{tikzcd}
		\cdots \arrow[r] & \redhom{H}_{n+1}(S^n) \arrow[r, "\iota_*"] & \redhom{H}_{n+1}(A) \arrow[r, "(\Sigma f)_*"] & \redhom{H}_{n+1}(A, S^n) \arrow[r, "\partial"] & \redhom{H}_n(S^n) \arrow[r, "\iota_*"] & \redhom{H}_{n}(A) \arrow[r] & \cdots
		\end{tikzcd}$$
		Given that $\redhom{H}_{n+1}(A,S^n) \cong \redhom{H}_{n+1}(A / S^n)$ and by the contractibility of $A$, this is equivalent to the exact sequence
		$$\begin{tikzcd}
		\cdots \arrow[r, "\iota_*"] & 0 \arrow[r, "(\Sigma f)_*"] & \redhom{H}_{n+1}(A / S^n) \arrow[r, "\partial"] & \redhom{H}_n(S^n) \arrow[r, "\iota_*"] & 0 \arrow[r] & \cdots
		\end{tikzcd}$$
		hence $\redhom{H}_{n+1}(A/S^n) \cong \redhom{H}_n(S^n)$ by the boundary map $\partial$.  Moreover, $A / S^n \cong S^{n+1}$, which suggests that $\redhom{H}_{n+1} (A/S^n) \cong \redhom{H}_{n+1}(S^{n+1})$.  By this and the naturality of the boundary maps in the $(A,S^n)$ pair long exact sequence, we have the following commutative diagram:
		$$\begin{tikzcd}
		\redhom{H}_{n+1}(S^{n+1}) \arrow[d, "(\Sigma f)_*" left] \arrow[r, "\partial", "\cong" below] & \redhom{H}_n(S^n) \arrow[d, "f_*"] \\
		\redhom{H}_{n+1}(S^{n+1}) \arrow[r, "\cong" above, "\partial" below] & \redhom{H}_n(S^n)
		\end{tikzcd}$$
		Hence we have that if $f_*$ is multiplication by $d$, then $(\Sigma f)_*$ is multiplication by $d$.  Therefore, $\deg \Sigma f = \deg f$.
		
		\argcheck{Did this a long time ago, should probably double-check to see if this argument is correct!}
		
	\end{proof}
	
	\clearpage

	% Alpha bibliography style.  Make second argument to thebigliography environment the largest largest explicit label used.
	\begin{thebibliography}{"Hatcher01"}
		\bibitem{hatcher01}
		Allen Hatcher,
		\emph{Algebraic Topology},
		Cambridge University Press,
		1st Edition,
		2001.	
	\end{thebibliography}

\end{document}
