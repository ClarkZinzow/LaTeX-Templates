% Basic lecture notes template.
%
% By Clark Zinzow

\documentclass[twoside]{amsart}

%  Make sure these parameters are set to what you want!
\newcommand{\authorName}{Clark Zinzow}  % Author's name.
\newcommand{\className}{Subject 101: Introduction to Intermediate}  % Course name.
\newcommand{\instName}{Professor Professorson}  % Instructor's name.
\newcommand{\semester}{Fall/Spring \the\year}  % Semester in which the class took place.
\newcommand{\universityName}{University of Wisconsin-Madison}
\newcommand{\authorEmail}{czinzow@wisc.edu}  % Author's e-mail.

%  New parameters, since we're dealing with a single lecture.
\newcommand{\lectureNum}{1}  % Lecture number.
\newcommand{\lectureTopic}{Topic of Lecture}  % Lecture topic.
\newcommand{\lectureDate}{October 29th, 2015}  % Lecture date.

%  Packages.
\usepackage{mathrsfs}
\usepackage{amssymb,amsmath,amsthm}
\usepackage{enumerate}
\usepackage[margin=1in]{geometry}
\usepackage{verbatim}
\usepackage[section]{placeins}
\usepackage{array}
\usepackage{nicefrac}
\usepackage{listings}
\usepackage{parskip}
\usepackage{graphicx}
\usepackage{placeins}
\usepackage{todonotes}

% Basic amsthm definitions

% Styles:
%   - plain:  italic text, extra space above and below;
%   - definition:  upright text, extra space above and below;
%   - remark:  upright text, no extra space above and below.

% plain style
\theoremstyle{plain} % default
\newtheorem{thm}{Theorem}[section]
\newtheorem{prop}[thm]{Proposition}
\newtheorem{conj}[thm]{Conjecture}
\newtheorem{crit}[thm]{Criterion}
\newtheorem{assrt}[thm]{Assertion}
\newtheorem{lem}[thm]{Lemma}
\newtheorem*{cor}{Corollary}
\newtheorem{fact}[thm]{Fact}

% definition style
\theoremstyle{definition}
\newtheorem{defn}{Definition}[section]
\newtheorem{condition}{Condition}[section]
\newtheorem{problem}{Problem}[section]
\newtheorem{axiom}{Axiom}[section]
\newtheorem{exmp}{Example}[section]
\newtheorem{property}{Property}[section]
\newtheorem{assump}{Assumption}[section]
\newtheorem{hypoth}{Hypothesis}[section]
\newtheorem{ques}{Question}[section]
\newtheorem{xca}[exmp]{Exercise}

% remark style
\theoremstyle{remark}
\newtheorem*{rem}{Remark}
\newtheorem*{note}{Note}
\newtheorem*{notation}{Notation}
\newtheorem*{clm}{Claim}
\newtheorem*{summary}{Summary}
\newtheorem*{acknowl}{Acknowledgment}
\newtheorem*{concl}{Conclusion}
\newtheorem*{case}{Case}

% Proof workaround:  The proof environment is defined as a LaTeX list, so a substitute ``name"
% that is longer than one output line will not break as it should.  The following is a
% preamble hack until amsthm fixes this.
\makeatletter
\renewenvironment{proof}[1][\proofname]{\par
	\pushQED{\qed}%
	\normalfont \topsep6\p@\@plus6\p@\relax
	\trivlist
	\item\relax
	{\itshape
		#1\@addpunct{.}}\hspace\labelsep\ignorespaces
}{%
\popQED\endtrivlist\@endpefalse
}
\makeatother

%  In text citations macro.
%  Usage:  \cite{***last name initial of each author, year published***}
%  Ex:  \cite{S96} for Peter Shor's 1996 paper "Polynomial-Time Algorithms for Prime
%         Factorization and Discrete Logarithms on a Quantum Computer"
%  Ex:  \cite{KMP77} for "Fast pattern matching in strings" by Donald Knuth, James H. Morris
%         and Vaughan Pratt.
%  Ex:  \cite{P03} for Grisha Perelman's 2003 paper "Finite extinction time for the solutions to
%         the Ricci flow on certain three-manifolds".
\renewcommand{\cite}[1]{[#1]}
\def\beginrefs{\begin{list}%
		{[\arabic{equation}]}{\usecounter{equation}
			\setlength{\leftmargin}{2.0truecm}\setlength{\labelsep}{0.4truecm}%
			\setlength{\labelwidth}{1.6truecm}}}
	\def\endrefs{\end{list}}
\def\bibentry#1{\item[\hbox{[#1]}]}

%  Suppresses indents.
\newlength{\tindent}
\setlength{\tindent}{\parindent}
\setlength{\parindent}{0pt}
\renewcommand{\indent}{\hspace*{\tindent}}

\title{Lectures Notes for \className}
\author{\authorName}
\email{\authorEmail}
\date{\today}

\begin{document}
	
	%\lstset{language=Matlab}
	\lstset{
		basicstyle=\ttfamily,
		columns=flexible,
		breaklines=true,
		numbers=none,
		frame=single
	}
	
	\begin{center}
		\framebox{
			\vbox{\vspace{2mm}
				\hbox to 6.28in { {\large \className
							\hfill \semester} }
				\vspace{4mm}
				\hbox to 6.28in { {\huge \bf \hfill Lecture \lectureNum: \lectureTopic  \hfill} }
				\vspace{4mm}
				\hbox to 6.28in { {\large \textbf{Lecturer:} \instName \hfill \textbf{Author:} \authorName} }
				\vspace{2mm}}
		}
	\end{center}
	\markboth{Lecture \lectureNum: \lectureTopic}{Lecture \lectureNum: \lectureTopic}
	
	\vspace*{4mm}
	
	Date of Lecture:  \lectureDate.
	
	\section{Subtopic Section 1}
	
	Text text text.  
	
	\begin{defn}
		Let the the \emph{intermediacy} of the topic be denoted by $\alpha$.
	\end{defn}
	
	\begin{rem}
		Note that ``intermediacy" is not a word.
	\end{rem}
	
	\begin{note}
		Note we are noting something in the above remark.  One could claim that it is actually a note.
	\end{note}
	
	Lorem ipsum whatever.
	
	\begin{thm}
		$x - y = 0 \iff y - x = 0$.
	\end{thm}
	\begin{proof}
		$x - y = 0 \iff x = y \iff y = x \iff y - x = 0$ by the reflexivity of equality.
	\end{proof}
	
	\subsection{Subsection 1}
	
	Text text text.
	
	\subsubsection{Subsubsection 1}
	
	Text text text.
	
	\subsubsection{Subsubsection 1}
	
	Text text text.
	
	\subsubsection{Subsubsection 1}
	
	Text text text.
	
	\subsection{Subsection 1}
	
	Text text text.
	
	\subsection{Subsection 1}
	
	Text text text.
	
	\section{Subtopic Section 1}
	
	Text text text.
	
	\subsection{Subsection 2}
	
	Text text text.
	
	\clearpage 
	
	\section{new stuff}
	
	\subsection{blah}
	
\end{document}
