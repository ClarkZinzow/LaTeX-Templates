% Basic lecture notes template.
%
% By Clark Zinzow

\documentclass[twoside, titlepage]{amsbook}

%  Make sure these parameters are set to what you want!
\newcommand{\authorName}{Clark Zinzow}  % Author's name.
\newcommand{\authorEmail}{czinzow@wisc.edu}  % Author's e-mail.
\newcommand{\subjectName}{Subject}  % Subject.
\newcommand{\institutionName}{University of Wisconsin-Madison}
\newcommand{\lastmodified}{\today}

% Basic amsthm definitions

% Styles:
%   - plain:  italic text, extra space above and below;
%   - definition:  upright text, extra space above and below;
%   - remark:  upright text, no extra space above and below.

% plain style
\theoremstyle{plain} % default
\newtheorem{thm}{Theorem}[section]
\newtheorem{prop}[thm]{Proposition}
\newtheorem{conj}[thm]{Conjecture}
\newtheorem{crit}[thm]{Criterion}
\newtheorem{assrt}[thm]{Assertion}
\newtheorem{lem}[thm]{Lemma}
\newtheorem*{cor}{Corollary}
\newtheorem{fact}[thm]{Fact}

% definition style
\theoremstyle{definition}
\newtheorem{defn}{Definition}[section]
\newtheorem{condition}{Condition}[section]
\newtheorem{problem}{Problem}[section]
\newtheorem{axiom}{Axiom}[section]
\newtheorem{exmp}{Example}[section]
\newtheorem{property}{Property}[section]
\newtheorem{assump}{Assumption}[section]
\newtheorem{hypoth}{Hypothesis}[section]
\newtheorem{ques}{Question}[section]
\newtheorem{xca}[exmp]{Exercise}

% remark style
\theoremstyle{remark}
\newtheorem*{rem}{Remark}
\newtheorem*{note}{Note}
\newtheorem*{clm}{Claim}
\newtheorem*{summary}{Summary}
\newtheorem*{acknowl}{Acknowledgments}
\newtheorem*{concl}{Conclusion}
\newtheorem*{case}{Case}

\newtheoremstyle{notation}
{}
{}
{}
{}
{\itshape}
{:}
{\newline}
{}
\theoremstyle{notation}
\newtheorem*{notation}{Notation}

% Proof workaround:  The proof environment is defined as a LaTeX list, so a 
% substitute ``name" that is longer than one output line will not break as it
% should.  The following is a preamble hack until amsthm fixes this.
\makeatletter
\renewenvironment{proof}[1][\proofname]{\par
	\pushQED{\qed}%
	\normalfont \topsep6\p@\@plus6\p@\relax
	\trivlist
	\item\relax
	{\itshape
		#1\@addpunct{.}}\hspace\labelsep\ignorespaces
}{%
\popQED\endtrivlist\@endpefalse
}
\makeatother

\usepackage{mathrsfs}
\usepackage{amssymb,amsmath,amsthm}
\usepackage{enumerate}
\usepackage[margin=1in]{geometry}
\usepackage{verbatim}
\usepackage[section]{placeins}
\usepackage{array}
\usepackage{nicefrac}
\usepackage{listings}
\usepackage{parskip}
\usepackage{graphicx}  % Note:  Use \afterpage{\clearpage} to flush all
%        processed floats, not \clearpage.
\usepackage[colorinlistoftodos]{todonotes}
\usepackage{tikz-cd}
\usepackage{fancyhdr}
\usepackage{bm}
\usepackage{dsfont}
\usepackage{mdwlist}
\usepackage{underscore}
\usepackage[colorlinks=true, citecolor=red, urlcolor=blue]{hyperref}

% Todo notes commands.
\newcommand{\unfinished}[1]{\todo[inline, color=red!40]{#1}}
\newcommand{\argcheck}[1]{\todo[inline, color=orange!40]{#1}}
\newcommand{\needprove}[1]{\todo[inline, color=violet!40]{#1}}
\newcommand{\detail}[1]{\todo[inline, color=blue!40]{#1}}
\newcommand{\insertref}[1]{\todo[inline, color=green!40]{#1}}
\newcommand{\badnotation}[1]{\todo[inline, color=yellow!40]{#1}}
\newcommand{\further}[1]{\todo[inline, color=cyan!40]{#1}}


%  In text citations macro.
%  Usage:  \cite{***last name initial of each author, year published***}
%  Ex:  \cite{S96} for Peter Shor's 1996 paper "Polynomial-Time Algorithms for Prime
%         Factorization and Discrete Logarithms on a Quantum Computer"
%  Ex:  \cite{KMP77} for "Fast pattern matching in strings" by Donald Knuth, James H. Morris
%         and Vaughan Pratt.
%  Ex:  \cite{P03} for Grisha Perelman's 2003 paper "Finite extinction time for the solutions to
%         the Ricci flow on certain three-manifolds".
\renewcommand{\cite}[1]{[#1]}
\def\beginrefs{\begin{list}%
		{[\arabic{equation}]}{\usecounter{equation}
			\setlength{\leftmargin}{2.0truecm}\setlength{\labelsep}{0.4truecm}%
			\setlength{\labelwidth}{1.6truecm}}}
	\def\endrefs{\end{list}}
\def\bibentry#1{\item[\hbox{[#1]}]}

%  Suppresses indents.
\newlength{\tindent}
\setlength{\tindent}{\parindent}
\setlength{\parindent}{0pt}
\renewcommand{\indent}{\hspace*{\tindent}}

\newcommand{\morph}{\longrightarrow}
\newcommand{\lmorph}{\longleftarrow}
\newcommand{\morpha}[1]{\overset{#1}{\morph}}
\newcommand{\lmorpha}[1]{\overset{#1}{\lmorph}}
\newcommand{\ses}[3]{%
	\begin{tikzcd}[ampersand replacement=\&]
		0 \arrow[r] \& #1 \arrow[r] \& #2 \arrow[r] \& #3 \arrow[r] \& 0
	\end{tikzcd}%
}
\newcommand{\sesm}[5]{%
	\begin{tikzcd}[ampersand replacement=\&]
		0 \arrow[r] \& #1 \arrow[r, "#4"] \& #2 \arrow[r, "#5"] \& #3 \arrow[r] \& 0
	\end{tikzcd}%
}
\newcommand{\suchthat}{\: \big\rvert \:}
\newcommand{\suchthatBig}{\: \Big\rvert \:}
\newcommand{\suchthathuge}{\: \huge\rvert \:}
\renewcommand{\Re}{\operatorname{Re}}
\renewcommand{\Im}{\operatorname{Im}}
\DeclareMathOperator*{\prob}{Pr}

% Subject specific commands.




\begin{document}
	
\begin{titlepage}
	\centering
	\vspace*{\baselineskip}
	\rule{\textwidth}{1.6pt}\vspace*{-\baselineskip}\vspace*{2pt}
	\rule{\textwidth}{0.4pt}\\[\baselineskip]
	{\Huge Notes on  \subjectName }
	\rule{\textwidth}{0.4pt}\vspace*{-\baselineskip}\vspace*{3pt}
	\rule{\textwidth}{1.6pt}\\[\baselineskip]
	\vspace*{4\baselineskip}
	\huge By \\[\baselineskip]
	{ \authorName}\\[\baselineskip]
	\vspace*{3\baselineskip}
	\huge \textbf{Last Modified:} { \lastmodified } \par
	\vspace*{3\baselineskip}
	\huge {\itshape \institutionName \par}
	\vfill
	{ If you find any errors in these notes, please e-mail me at \authorEmail. \par}
\end{titlepage}

\tableofcontents

\chapter*{Preface}

	\section*{Notation:}
		We use the following non-standard notation:
		\begin{itemize}
			\item 
		\end{itemize}
	
	\section*{Acknowledgments:}
		Thank you to the following people for helping me, in some way, with these solutions:
		\begin{itemize}
			\item 
		\end{itemize}

\part{First Part of Subject}

	\chapter{Introduction}
	
		\section{Section 1}
		
			Text text text.  
			
			\begin{defn}
				Let the the \emph{intermediacy} of the topic be denoted by $\alpha$.
			\end{defn}
			
			\begin{rem}
				Note that ``intermediacy" is not a word.
			\end{rem}
			
			\begin{note}
				Note we are noting something in the above remark.  One could claim that it is actually a note.
			\end{note}
			
			Lorem ipsum whatever.
			
			\begin{thm}
				$x - y = 0 \iff y - x = 0$.
			\end{thm}
			\begin{proof}
				$x - y = 0 \iff x = y \iff y = x \iff y - x = 0$ by the reflexivity of equality.
			\end{proof}
			
			\subsection{Subsection 1}
			
				Text text text.
			
				\subsubsection{Subsubsection 1}
				
					Text text text.
			
				\subsubsection{Subsubsection 2}
				
					Text text text.
			
				\subsubsection{Subsubsection 3}
				
					Text text text.
			
			\subsection{Subsection 2}
			
				Text text text.
			
			\subsection{Subsection 3}
			
				Text text text.
		
		\section{Section 2}
		
			Text text text.
			
			\subsection{Subsection 2}
			
				Text text text.
		
		\section{Section 3}
		
			Text text text.  
			
			\begin{defn}
				Let the the \emph{intermediacy} of the topic be denoted by $\alpha$.
			\end{defn}
			
			\begin{rem}
				Note that ``intermediacy" is not a word.
			\end{rem}
			
			\begin{note}
				Note we are noting something in the above remark.  One could claim that it is actually a note.
			\end{note}
			
			Lorem ipsum whatever.
			
			\begin{thm}
				$x - y = 0 \iff y - x = 0$.
			\end{thm}
			\begin{proof}
				$x - y = 0 \iff x = y \iff y = x \iff y - x = 0$ by the reflexivity of equality.
			\end{proof}
			
			\subsection{Subsection 1}
			
				Text text text.
				
				\subsubsection{Subsubsection 1}
				
					Text text text.
				
				\subsubsection{Subsubsection 2}
				
					Text text text.
				
				\subsubsection{Subsubsection 3}
				
					Text text text.
			
			\subsection{Subsection 2}
			
				Text text text.
			
			\subsection{Subsection 3}
			
				Text text text.
		
		\section{Section 4}
		
			Text text text.
			
			\subsection{Subsection 1}
			
				Text text text.
	
	\chapter{More stuff}
	
		\section{Section 1}
		
			

\part{Second Part of Subject}

	\chapter{Etc. etc. etc.}
	
		
	
\clearpage

% Alpha bibliography style.  Make second argument to thebigliography environment the largest largest explicit label used.
\begin{thebibliography}{"Fol99"}
	\bibitem[Fol99]{folland99}
	Gerald B. Folland,
	\emph{Real Analysis: Modern Techniques and Their Applications},
	Wiley,
	2nd Edition,
	1999.	
\end{thebibliography}

\end{document}
