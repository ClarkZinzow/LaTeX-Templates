% Basic lecture notes template.
%
% By Clark Zinzow

\documentclass[twoside, titlepage]{amsart}

%  Make sure these parameters are set to what you want!
\newcommand{\authorName}{Clark Zinzow}  % Author's name.
\newcommand{\subject}{Subject}  % Subject.
\newcommand{\courseNumber}{101}  % Course number.
\newcommand{\courseName}{Introduction to Intermediate}  % Course name.
\newcommand{\instName}{Professor Professorson}  % Instructor's name.
\newcommand{\semester}{Fall/Spring \the\year}  % Semester class taken.
\newcommand{\universityName}{University of Wisconsin-Madison}  % University.
\newcommand{\authorEmail}{czinzow@wisc.edu}  % Author's e-mail.

%  Basic amsthm definitions

%  Styles:
%   - plain:  italic text, extra space above and below;
%   - definition:  upright text, extra space above and below;
%   - remark:  upright text, no extra space above and below.

%  Plain style
\theoremstyle{plain} % default
\newtheorem{thm}{Theorem}[section]
\newtheorem{prop}[thm]{Proposition}
\newtheorem{conj}[thm]{Conjecture}
\newtheorem{crit}[thm]{Criterion}
\newtheorem{assrt}[thm]{Assertion}
\newtheorem{lem}[thm]{Lemma}
\newtheorem*{cor}{Corollary}
\newtheorem{fact}[thm]{Fact}

%  Definition style
\theoremstyle{definition}
\newtheorem{defn}{Definition}[section]
\newtheorem{condition}{Condition}[section]
\newtheorem{problem}{Problem}[section]
\newtheorem{axiom}{Axiom}[section]
\newtheorem{exmp}{Example}[section]
\newtheorem{property}{Property}[section]
\newtheorem{assump}{Assumption}[section]
\newtheorem{hypoth}{Hypothesis}[section]
\newtheorem{ques}{Question}[section]
\newtheorem{xca}[exmp]{Exercise}

%  Remark style
\theoremstyle{remark}
\newtheorem*{rem}{Remark}
\newtheorem*{note}{Note}
\newtheorem*{clm}{Claim}
\newtheorem*{summary}{Summary}
\newtheorem*{acknowl}{Acknowledgments}
\newtheorem*{concl}{Conclusion}
\newtheorem*{case}{Case}

\newtheoremstyle{notation}
{}
{}
{}
{}
{\itshape}
{:}
{\newline}
{}
\theoremstyle{notation}
\newtheorem*{notation}{Notation}

%  Proof workaround:  The proof environment is defined as a LaTeX list, so a 
%  substitute ``name" that is longer than one output line will not break as it
%  should.  The following is a preamble hack until amsthm fixes this.
\makeatletter
\renewenvironment{proof}[1][\proofname]{\par
	\pushQED{\qed}%
	\normalfont \topsep6\p@\@plus6\p@\relax
	\trivlist
	\item\relax
	{\itshape
		#1\@addpunct{.}}\hspace\labelsep\ignorespaces
}{%
\popQED\endtrivlist\@endpefalse
}
\makeatother

%  Packages
\usepackage{mathrsfs}
\usepackage{amssymb,amsmath,amsthm}
\usepackage{enumerate}
\usepackage[margin=1in]{geometry}
\usepackage{verbatim}
\usepackage[section]{placeins}
\usepackage{array}
\usepackage{nicefrac}
\usepackage{parskip}
\usepackage{graphicx}  % Note:  Use \afterpage{\clearpage} to flush all
                       %        processed floats, not \clearpage.
\usepackage[colorinlistoftodos]{todonotes}
\usepackage{fancyhdr}
\usepackage{dsfont}
\usepackage{mdwlist}
\usepackage{underscore}
\usepackage[colorlinks=true, citecolor=red, urlcolor=blue]{hyperref}

%  Subject specific packages
%\usepackage{tikz-cd}  % Great for commutative diagrams.
%\usepackage{bm}  % Great for bold vectors.
\usepackage{listings}  % Great for source code.

%  Todo notes commands, color-coded to type of todo.
\newcommand{\unfinished}[1]{\todo[inline, color=red!40]{#1}}
\newcommand{\argcheck}[1]{\todo[inline, color=orange!40]{#1}}
\newcommand{\needprove}[1]{\todo[inline, color=violet!40]{#1}}
\newcommand{\detail}[1]{\todo[inline, color=blue!40]{#1}}
\newcommand{\insertref}[1]{\todo[inline, color=green!40]{#1}}
\newcommand{\badnotation}[1]{\todo[inline, color=yellow!40]{#1}}
\newcommand{\further}[1]{\todo[inline, color=cyan!40]{#1}}

%  Lecture numbering.
\newcounter{lecnum}
\renewcommand{\thepage}{\thelecnum-\arabic{page}}
\renewcommand{\thesection}{\thelecnum.\arabic{section}}
\renewcommand{\theequation}{\thelecnum.\arabic{equation}}
\renewcommand{\thefigure}{\thelecnum.\arabic{figure}}
\renewcommand{\thetable}{\thelecnum.\arabic{table}}

%  Lecture header macro.
%  Usage:  \lecture{lecture number}{topic}{date}
\newcommand{\lecture}[3]{
	\pagestyle{myheadings}
	\thispagestyle{plain}
	\newpage
	\setcounter{lecnum}{#1}
	\setcounter{page}{1}
	\noindent
	\begin{center}
		\framebox{
			\vbox{\vspace{2mm}
				\hbox to 6.28in { {\large \subject\ \courseNumber: \courseName
						\hfill #3} }
				\vspace{4mm}
				\hbox to 6.28in { {\huge \hfill \textbf{Lecture #1:} \parbox{4in}{\centering \textbf{#2}} \hfill } }
				\vspace{4mm}
				\hbox to 6.28in { { \large \textbf{Lecturer:} \instName \hfill \textbf{Author:} \authorName} }
				\vspace{2mm}}
		}
	\end{center}
	\markboth{Lecture #1: #2}{Lecture #1: #2}

	\vspace*{4mm}
}

%  Suppresses indents.
\newlength{\tindent}
\setlength{\tindent}{\parindent}
\setlength{\parindent}{0pt}
\renewcommand{\indent}{\hspace*{\tindent}}

%  Such that symbol commands.
\newcommand{\suchthat}{\: \big\rvert \:}
\newcommand{\suchthatBig}{\: \Big\rvert \:}
\newcommand{\suchthathuge}{\: \huge\rvert \:}

%  Subject specific commands.

%\renewcommand{\Re}{\operatorname{Re}}
%\renewcommand{\Im}{\operatorname{Im}}
%\DeclareMathOperator*{\prob}{Pr}

\begin{document}
	
	\begin{titlepage}
		\centering
		\vspace*{\baselineskip}
		\rule{\textwidth}{1.6pt}\vspace*{-\baselineskip}\vspace*{2pt}
		\rule{\textwidth}{0.4pt}\\[\baselineskip]
		\Huge Lecture Notes\\ for \\
			\subject\ \courseNumber: \courseName
		\\
		\rule{\textwidth}{0.4pt}\vspace*{-\baselineskip}\vspace*{3pt}
		\rule{\textwidth}{1.6pt}\\[\baselineskip]
		\vspace*{3\baselineskip}
		\huge {\itshape \universityName \par}
		\vspace*{\baselineskip}
		\semester \par
		\vspace*{3\baselineskip}
		\begin{minipage}{0.4\textwidth}
			\begin{flushleft}
				\begin{center}
					\huge \textbf{Instructor:}  \\[\baselineskip]
					{ \instName}	
				\end{center}		
			\end{flushleft}
		\end{minipage}
		\begin{minipage}{0.4\textwidth}
			\begin{flushright}
				\begin{center}
					\huge \textbf{Author:} \\[\baselineskip]
					{ \authorName}
				\end{center}
			\end{flushright}
		\end{minipage}\\
		\vfill
		%  Contact if errors are found disclaimer.
		{ If you find any errors in these notes, please e-mail me at \authorEmail. \par}
	\end{titlepage}
	
	This is a set of lecture notes for \subject\ \courseNumber: \courseName, taught at \universityName\ by \instName\ in \semester.
	\hspace{0pt} \\
	
	\section*{Notation:}
	We use the following non-standard notation:
	\begin{itemize}
		\item 
	\end{itemize}
	
	\section*{Acknowledgments:}
	Thank you to the following people for helping me, in some way, with these lecture notes:
	\begin{itemize}
		\item 
	\end{itemize}
	
	\lecture{1}{Topic of First Lecture}{\today}
	
	\section{Subtopic Section 1}
	
	Text text text.  
	
	\begin{defn}
		Let the the \emph{intermediacy} of the topic be denoted by $\alpha$.
	\end{defn}
	
	\begin{rem}
		Note that ``intermediacy" is not a word.
	\end{rem}
	
	\begin{note}
		Note we are noting something in the above remark.  One could claim that it is actually a note.
	\end{note}
	
	Lorem ipsum whatever.
	
	\begin{thm}
		$x - y = 0 \iff y - x = 0$.
	\end{thm}
	\begin{proof}
		$x - y = 0 \iff x = y \iff y = x \iff y - x = 0$ by the reflexivity of equality.
	\end{proof}
	
	\subsection{Subsection 1}
	
	Text text text.
	
	\subsubsection{Subsubsection 1}
	
	Text text text.
	
	\subsubsection{Subsubsection 1}
	
	Text text text.
	
	\subsubsection{Subsubsection 1}
	
	Text text text.
	
	\subsection{Subsection 1}
	
	Text text text.
	
	\subsection{Subsection 1}
	
	Text text text.
	
	\section{Subtopic Section 1}
	
	Text text text.
	
	\subsection{Subsection 2}
	
	Text text text.
	
	\clearpage 
	
	\section{More stuff!}
	
	\subsection{blah}
	
	Neat-o!
	
	\begin{itemize}
		\item Oh no.
		\item You're live-TeXing your lecture notes.
		\item The professor is talking really fast.
		\item No time to organize your notes into neat sections, subsections, and subsubsections!
		\item Good thing you can just create an unordered list like this...
		\begin{itemize}
			\item and have nested lists as well!
			\item Look at that hierarchical structure!
		\end{itemize}
		\item I'd say that's pretty neat.
	\end{itemize}
			
	\clearpage
	
	\lecture{2}{Topic of Second Lecture}{\today}
	
	\section{Subtopic Section 1}
	
	Text text text.  
	
	\begin{defn}
		Let the the \emph{intermediacy} of the topic be denoted by $\alpha$.
	\end{defn}
	
	\begin{rem}
		Note that ``intermediacy" is not a word.
	\end{rem}
	
	\begin{note}
		Note we are noting something in the above remark.  One could claim that it is actually a note.
	\end{note}
	
	Lorem ipsum whatever.
	
	\begin{thm}
		$x - y = 0 \iff y - x = 0$.
	\end{thm}
	\begin{proof}
		$x - y = 0 \iff x = y \iff y = x \iff y - x = 0$ by the reflexivity of equality.
	\end{proof}
	
	\subsection{Subsection 1}
	
	Text text text.
	
	\subsubsection{Subsubsection 1}
	
	Text text text.
	
	\subsubsection{Subsubsection 1}
	
	Text text text.
	
	\subsubsection{Subsubsection 1}
	
	Text text text.
	
	\subsection{Subsection 1}
	
	Text text text.
	
	\subsection{Subsection 1}
	
	Text text text.
	
	\section{Subtopic Section 1}
	
	Text text text.
	
	\subsection{Subsection 2}
	
	Text text text.
	
	\clearpage 
	
	\section{More stuff!}
	
	\subsection{blah}
	
	Neat-o!
	
	\begin{itemize}
		\item Oh no.
		\item You're live-TeXing your lecture notes.
		\item The professor is talking really fast.
		\item No time to organize your notes into neat sections, subsections, and subsubsections!
		\item Good thing you can just create an unordered list like this...
		\begin{itemize}
			\item and have nested lists as well!
			\item Look at that hierarchical structure!
		\end{itemize}
		\item I'd say that's pretty neat.
	\end{itemize}
	
	\clearpage
	
	% Alpha bibliography style.  Make second argument to thebigliography environment the largest largest explicit label used.
	\begin{thebibliography}{"McAuthorface16"}
		\bibitem{realbook16}
		Writey McAuthorface,
		\emph{Introduction to Advanced Intermediate: For Beginning Experts},
		Publishy McPressface,
		10th Edition,
		2016.
	\end{thebibliography}

\end{document}
